\section*{ВВЕДЕНИЕ}
\addcontentsline{toc}{section}{ВВЕДЕНИЕ}

Компилятор -- программное обеспечение, которое переводит поданный на вход текст программы, написанный на одном из языков программирования -- исходном, в машинный код для исполнения на компьютере. В процессе преобразования команд выполняется оптимизация кода и анализ ошибок, что позволяет улучшить производительность и избежать некоторых сбоев при выполнении программы. \cite{bib:compilerIS}

Целью данной курсовой работы является разработка компилятора языка Oberon. Компилятор должен выполнять чтение текстового файла, содержащего код на языке Oberon и генерировать на выходе программу, пригодную для запуска.

Для достижения цели необходимо решить следующие задачи:
\begin{itemize}	
	\item проанализировать грамматику языка Oberon;
	
	\item изучить существующие средства для анализа исходных кодов программ, системы для генерации низкоуровневого кода, запуск которого возможен на большинстве из используемых платформ и операционных систем;
	
	\item реализовать прототип компилятора.
\end{itemize}


\pagebreak




















